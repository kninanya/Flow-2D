\documentclass{article} % Don't change this
\usepackage[a4paper, left=2.5cm, right=2cm, top=2cm, bottom=2cm]{geometry}
\usepackage[brazil]{babel}
\usepackage[utf8]{inputenc}

\newcommand{\trinum}[1]{%
	\triangle\hspace{-.57em}\raisebox{0.1em}{\scalebox{.5}{#1}}
}
\usepackage{lscape}
\usepackage{amsmath}
\usepackage{amsthm}
\usepackage{amsfonts}
\usepackage{amssymb}
\usepackage[usenames,dvipsnames]{xcolor}
\usepackage{graphicx}
\usepackage[siunitx]{circuitikz}
\usepackage{tikz}
\usepackage[colorinlistoftodos, color=orange!50]{todonotes}
\usepackage{hyperref}
%\usepackage[numbers, square]{natbib}
\usepackage{fancybox}
\usepackage{epsfig}
\usepackage{soul}
\usepackage[framemethod=tikz]{mdframed}
\usepackage{multirow}

\usepackage{gensymb}
\setcounter{MaxMatrixCols}{20}

\usepackage{paralist} %to enable {inparaenum}
\usepackage{natbib} %Natual bibliography
\usepackage{graphicx}
\usepackage{float} %To tables and figures
\usepackage{caption} %To describe figures
\usepackage{graphicx}
\usepackage{refstyle}
%\usepackage[latin1]{inputenc} %Type of decodification -Latinunderstandsportuguese acents
\usepackage{subcaption} % group of figures
%\usepackage{amsmath} %to enumerate an equantion according to section
\usepackage{booktabs} %to create tables
\usepackage{graphicx}
\usepackage{wrapfig}
\usepackage{amsmath}
\usepackage{multirow}
\usepackage{hyperref}%to show labels in red,blue and green colors
%\numberwithin{equation}{section} %to enumerate an equantion according to section
%\numberwithin{figure}{section} %to enumerate a figure according to section





\newcommand{\blah}{blah blah blah \dots}



\setlength{\marginparwidth}{3.4cm}

% NEW COUNTERS
\newcounter{points}
\setcounter{points}{100}
\newcounter{spelling}
\newcounter{usage}
\newcounter{units}
\newcounter{other}
\newcounter{source}
\newcounter{concept}
\newcounter{missing}
\newcounter{math}

% COMMANDS
%\newcommand{\raisa}[2]{\colorbox{Yellow}{#1} \todo{#2}}
\newcommand{\arbitrary}[2]{\todo{#1 #2} \addtocounter{points}{#2} \addtocounter{other}{#2}}
\newcommand{\english}{\todo{LANGUAGE (-1)} \addtocounter{points}{-1}
	\addtocounter{usage}{-1}}
\newcommand{\units}{\todo{UNITS (-1)} \addtocounter{points}{-1}
	\addtocounter{units}{-1}}
\newcommand{\spelling}{\todo{SPELLING and GRAMMAR (-1)} \addtocounter{points}{-1}
	\addtocounter{spelling}{-1}}
\newcommand{\source}{\todo{SOURCE(S) (-2)} \addtocounter{points}{-2}
	\addtocounter{source}{-2}}
\newcommand{\concept}{\todo{CONCEPT (-2)} \addtocounter{points}{-2}
	\addtocounter{concept}{-2}}
\newcommand{\missing}[2]{\todo{MISSING CONTENT (#1) #2} \addtocounter{points}{#1}
	\addtocounter{missing}{#1}}
\newcommand{\maths}{\todo{MATH (-1)} \addtocounter{points}{-1}
	\addtocounter{math}{-1}}

\newcommand{\summary}[1]{
	\begin{mdframed}[nobreak=true]
		\begin{minipage}{\textwidth}
			\vspace{0.5cm}
			\begin{center}
				\Large{Grade Summary} \hrule 
			\end{center} \vspace{0.5cm}
			General Comments: #1
			
			\vspace{0.5cm}
			Possible Points \dotfill 100 \\
			Points Lost (Spelling and Grammar) \dotfill \thespelling \\
			Points Lost (Language) \dotfill \theusage \\
			Points Lost (Units) \dotfill \theunits \\
			Points Lost (Math) \dotfill \themath \\
			Points Lost (Sources) \dotfill \thesource \\
			Points Lost (Concept) \dotfill \theconcept \\
			Points Lost (Missing Content) \dotfill \themissing \\
			Other \dotfill \theother \\[0.5cm]
			\begin{center}
				\large{\textbf{Grade:} \fbox{\thepoints}}
			\end{center}
		\end{minipage}
\end{mdframed}}


\renewcommand*{\thefootnote}{\fnsymbol{footnote}}


\title{
	\normalfont \normalsize 
	\textsc{Pontificia Universidade Católica do Rio de Janeiro, RJ, Brasil \\ 
		Departamento de Engenharia Civil e Ambiental, Geotecnia} \\
	[10pt] 
	\rule{\linewidth}{0.5pt} \\[6pt] 
	\huge LISTA Nº 4\\
	\rule{\linewidth}{2pt}  \\[10pt]
}
\author{Karen Ninanya}
\date{\normalsize (1812565)}

\begin{document}
	
	\maketitle
	\noindent
	Professor \dotfill Celso Romanel\\
	Disciplina \dotfill CIV 2532 - Métodos Numéricos em Engenharia Civil\\
	Data \dotfill 15 de Outobro, 2018 \\
	
	\newpage
	%\tableofcontents
	\newpage


\section*{Questão}

\vspace{10mm}
Pede-se determinar pelo método dos elementos finitos, considerando elementos de três nós (T3), as cargas hidráulicas nodais e velocidades de fluxo nos elementos da malha da Fig. \ref{questao}. Admitir um coeficiente de permeabilidade isotrópico do solo de fundação \(k=1x10^{-4}\) cm/s.\\
\\
Considerar a origem dos eixes cartesianos no nó inferior esquerdo. Condições de contorno em relação ao nivel de referência na interface solo-rocha impermeável:\\
\\
\indent a) Linha equipotencial máxima \(h=13\) m.\\
\indent b) Contorno vertical esquerdo, admitindo ausência de fluxo \(h=13\) m.\\
\indent c)Linha equipotencial sob o centro da barragem, devido à simetria do problema \(h=10,5\) m.\\
\\
Os elementos T3 da malha são todos iguais, representados por triângulos retângulos com catetos de \(4\) m.

\begin{figure}[H]
	\centering
	\caption{Esquema geral do problema}
	\includegraphics[width=0.45\linewidth]{principal}	
	\label{questao}	
\end{figure}


\newpage

\section*{Solução}

\vspace{10mm}


\underline{\large \textit{Método dos elementos finitos}}\\

\begin{itemize}
	\item Discretização
\end{itemize}

A discretização do medio poroso foi feita a partir de elementos de 3 nós (T3) , como pode ser observado na figura \ref{global}. A vista local do elemento é mostrado na figura \ref{local}.

\begin{figure}[H]
	\centering
	\caption{Vista global do problema com elementos de três nós (T3)}	\includegraphics[width=0.65\linewidth]{elemento1}	
	\label{global}	
\end{figure}
\begin{figure}[H]
	\centering
	\caption{Vista do elemento T3}	\includegraphics[width=0.45\linewidth]{local}	
	\label{local}	
\end{figure}
As coordenadas dos nós de acordo à figura \ref{global} são mostradas na seguinte tabela.
\begin{table}[H]
	\centering
	\begin{tabular}{@{}lcccccccccccc@{}}
		\toprule
		\multirow{2}{*}{Coordenadas} & \multicolumn{12}{c}{Nós} \\ \cmidrule(l){2-13} 
		& 1 & 2 & 3 & 4 & 5 & 6 & 7 & 8 & 9 & 10 & 11 & 12 \\ \midrule
		x (m) & 0 & 4 & 8 & 12 & 0 & 4 & 8 & 12 & 0 & 4 & 8 & 12 \\
		y (m) & 0 & 0 & 0 & 0 & 4 & 4 & 4 & 4 & 8 & 8 & 8 & 8 \\ \bottomrule
	\end{tabular}
\end{table}


Sendo a equação de elementos finitos:

\begin{equation}
\Omega=\frac{1}{2}\int_{A}[q]^T[B]^T[R][B][q]dA-\int_{A}\bar{Q}[q]^T[N]^TdA-\int_{S_1}\bar{q}[q]^T[N]^TdS
\end{equation}

Ao derivar respecto de \([q]\), temos
\begin{equation}
\frac{\partial\Omega}{\partial [q]}=0=\int_{A}[B]^T[R][B][q]dA-\int_{A}\bar{Q}[N]^TdA-\int_{S_1}\bar{q}[N]^TdS
\end{equation}

\begin{equation}\label{geral}
\left[\int_{A}[B]^T[R][B]dA\right][q]=\left[\int_{A}\bar{Q}[N]^TdA+\int_{S_1}\bar{q}[N]^TdS\right]
\end{equation}

Agrupando.
\begin{equation}
[k][q]=[Q]
\end{equation}


\begin{itemize}
	\item Matriz elemental \([k]\)	
\end{itemize}

Da equação \ref{geral}, temos que a matriz elemental \([k]\) é a seguinte.
\begin{equation}
[k]=\int_{A}[B]^T[R][B]dA
\end{equation}
\begin{equation}\label{kelemento}
[k]=A[B]^T[R][B]
\end{equation}
onde o cálculo das matrices \([R]\)  e \([B]\) é da seguinte manera.



\begin{equation}
[R]=\begin{bmatrix}
k_x&0\\0&k_y
\end{bmatrix}
\end{equation}

\begin{equation}
[B]=\frac{1}{2A}\begin{bmatrix}
y_{23}&y_{31}&y_{12}\\x_{32}&x_{13}&x_{21}
\end{bmatrix}
\end{equation}

sendo o calculo das áreas (\(A\)) da seguinte forma.

\begin{equation}
A=\frac{1}{2}\biggr|\color{white}\begin{bmatrix}\color{black}
1&\color{black}x_1&\color{black}y_1\\\color{black}1&\color{black}x_2&\color{black}y_2\\\color{black}1&\color{black}x_3&\color{black}y_3
\color{white}\end{bmatrix}\color{black}\biggr|
\end{equation}



Então, a matriz \([k]\) elemental ficaria:
\begin{equation}
[k]=\frac{1}{2\hspace{4pt}\biggr|\color{white}\begin{bmatrix}\color{black}
	1&\color{black}x_1&\color{black}y_1\\\color{black}1&\color{black}x_2&\color{black}y_2\\\color{black}1&\color{black}x_3&\color{black}y_3
	\color{white}\end{bmatrix}\color{black}\biggr|}\begin{bmatrix}
y_{23}&x_{32}\\y_{31}&x_{13}\\y_{12}&x_{21}
\end{bmatrix}\begin{bmatrix}
k_x&0\\0&k_y
\end{bmatrix}\begin{bmatrix}
y_{23}&y_{31}&y_{12}\\x_{32}&x_{13}&x_{21}
\end{bmatrix}
\end{equation}
onde \(k_x=k_y=k=1x10^{-6}m/s\)\\
Por tanto as matrices \([k]\) elementares do problema são os seguintes.


\begin{equation}
[k]^{1,3,5,7,9,11}=\frac{k}{2}\begin{bmatrix}
2&-1&-1\\-1&1&0\\-1&0&1
\end{bmatrix}
\end{equation}
\begin{equation}
[k]^{2,4,6,8,10,12}=\frac{k}{2}\begin{bmatrix}
1&-1&0\\-1&2&-1\\0&-1&1
\end{bmatrix}
\end{equation}


\begin{itemize}
	\item Vetor \([Q]\)	elementar
\end{itemize}


Logo, o vetor \([Q]\) de acordo a equação \ref{geral} e o seguinte.
\begin{equation*}
[{Q}]=\int_{A}\bar{Q}[N]^TdA+\int_{S_1}\bar{q}[N]^TdS
\end{equation*}

Sendo o valor do fluxo retirao ou injetado \(\bar{Q}=0\) e o fluxo preescrito \(\bar{q}=0\) os vetores \([Q]\) elementares são os mostrados a continuação.
\begin{equation*}
[{Q}]^{1}=[{Q}]^{2}=[{Q}]^{3}=[{Q}]^{4}=[{Q}]^{5}=[{Q}]^{6}=[{Q}]^{7}=[{Q}]^{8}=[{Q}]^{9}=[{Q}]^{10}=[{Q}]^{11}=[{Q}]^{12}=\begin{bmatrix}
0\\0\\0
\end{bmatrix}
\end{equation*}
\begin{itemize}
	\item Matriz de correspondencia Global-local
\end{itemize}



\begin{table}[H]
	\centering
	\begin{tabular}{@{}ccccccccccccc@{}}
		\toprule
		\multirow{2}{*}{Global} & \multicolumn{12}{c}{Local} \\ \cmidrule(l){2-13} 
		& $\trinum{1}$& $\trinum{2}$ & $\trinum{3}$ & $\trinum{4}$ &$\trinum{5}$ &$\trinum{6}$ &$\trinum{7}$ &$\trinum{8}$ &$\trinum{9}$ &$\trinum{10}$ &$\trinum{11}$ &$\trinum{12}$ \\ \midrule
		1 & 1 & - & - & - & - & - & - & -& - & - & - & -\\
		2 & 2 & 1 & 1& -& - & - & - & -& - & - & - & - \\
		3 & - & - & 2 & 1& 1 & - & - & -& - & - & - & - \\
		4 & - & -& - & - & 2 & 1& - & -& - & - & - & -\\
		5 & 3 & 3 & - & -& - & - & 1 & -& - & - & - & - \\
		6 & - & 2 & 3 & 3& - & - & 2 & 1& 1 & - & - & -\\
		7 & - & - & - & 2 & 3 & 3 & - & -& 2 & 1 & 1 & -\\
		8 & - & - & - & -& - & 2 & - & -& -& - & 2 & 1 \\
		9 & - & - & - & - & - & - & 3 & 3&- & - & - & -\\
		10 & - & - & - & - & -& - & - & 2& 3 & 3 & - & -\\
		11 & - & - & - & - & -& - & - & -&  & 2 & 3 & 3\\
		12 & - & - & - & - & - & - & - & -& - & - & - & 2\\ \bottomrule
	\end{tabular}
\end{table}



\begin{itemize}
	\item Montagem da matriz global
\end{itemize}


\begin{equation}
\frac{k}{2}\begin{bmatrix}
\color{red}2&\color{red}-1&\color{red}0&\color{red}0&\color{red}-1&\color{red}0&\color{red}0&\color{red}0&\color{red}0&\color{red}0&\color{red}0&\color{red}0\\
\color{red}-1&4&-1&\color{red}0&\color{red}0&-2&0&\color{red}0&\color{red}0&\color{red}0&\color{red}0&\color{red}0\\
\color{red}0&-1&4&\color{red}-1&\color{red}0&0&-2&\color{red}0&\color{red}0&\color{red}0&\color{red}0&\color{red}0\\
\color{red}0&\color{red}0&\color{red}-1&\color{red}2&\color{red}0&\color{red}0&\color{red}0&\color{red}-1&\color{red}0&\color{red}0&\color{red}0&\color{red}0\\
\color{red}-1&\color{red}0&\color{red}0&\color{red}0&\color{red}4&\color{red}-2&\color{red}0&\color{red}0&\color{red}-1&\color{red}0&\color{red}0&\color{red}0\\
\color{red}0&-2&0&\color{red}0&\color{red}-2&8&-2&\color{red}0&\color{red}0&\color{red}-2&\color{red}0&\color{red}0\\
\color{red}0&0&-2&\color{red}0&\color{red}0&-2&8&\color{red}-2&\color{red}0&\color{red}0&\color{red}-2&\color{red}0\\
\color{red}0&\color{red}0&\color{red}0&\color{red}-1&\color{red}0&\color{red}0&\color{red}-2&\color{red}4&\color{red}0&\color{red}0&\color{red}0&\color{red}-1\\
\color{red}0&\color{red}0&\color{red}0&\color{red}0&\color{red}-1&\color{red}0&\color{red}0&\color{red}0&\color{red}2&\color{red}-1&\color{red}0&\color{red}0\\
\color{red}0&\color{red}0&\color{red}0&\color{red}0&\color{red}0&\color{red}-2&\color{red}0&\color{red}0&\color{red}-1&\color{red}4&\color{red}-1&\color{red}0\\
\color{red}0&\color{red}0&\color{red}0&\color{red}0&\color{red}0&\color{red}0&\color{red}-2&\color{red}0&\color{red}0&\color{red}-1&\color{red}4&\color{red}-1\\
\color{red}0&\color{red}0&\color{red}0&\color{red}0&\color{red}0&\color{red}0&\color{red}0&\color{red}-1&\color{red}0&\color{red}0&\color{red}-1&\color{red}2
\end{bmatrix}\begin{bmatrix}
\color{red}h_1\\h_2\\h_3\\\color{red}h_4\\\color{red}h_5\\h_6\\h_7\\\color{red}h_8\\\color{red}h_9\\\color{red}h_{10}\\\color{red}h_{11}\\\color{red}h_{12}\\
\end{bmatrix}=\begin{bmatrix}
\color{red}0\\0\\0\\\color{red}0\\\color{red}0\\0\\0\\\color{red}0\\\color{red}0\\\color{red}0\\\color{red}0\\\color{red}0
\end{bmatrix}
\end{equation}
\begin{itemize}
	\item Introdução das condições de contorno
\end{itemize}

Sendo o valor das cargas hidráulicas \(h_1\), \(h_4\), \(h_5\), \(h_8\), \(h_9\), \(h_{10}\), \(h_{11}\) e \(h_{12}\) valores conhecidos o sistema linear ficou da seguinte forma.


\begin{equation}
\begin{bmatrix}
4&-1&-2&0\\
-1&4&0&-2\\
-2&0&8&-2\\
0&-2&-2&8
\end{bmatrix}\begin{bmatrix}
h_2\\h_3\\h_6\\h_7
\end{bmatrix}=\begin{bmatrix}
h_1\\h_4\\2(h_5+h_{10})\\2(h_8+h_{11})
\end{bmatrix}=\begin{bmatrix}
13,00\\10,50\\52,00\\47,00
\end{bmatrix}
\end{equation}

Ao resolver o sistema linear no MatLab os valores das cargas hidraulicas \(h_2\), \(h_3\), \(h_6\) e \(h_7\) forem obtidos.
\begin{equation}
\begin{bmatrix}
h_1\\h_2\\h_3\\h_4\\h_5\\h_6\\h_7\\h_8\\h_9\\h_{10}\\h_{11}\\h_{12}
\end{bmatrix}=\begin{bmatrix}
13,00\\12,48\\11,72\\10,50\\13,00\\12,61\\11,96\\10,50\\13,00\\13,00\\13,00\\10,50
\end{bmatrix}m
\end{equation}

Na figura \ref{result} são mostradas as cargas hidráulicas nodais.
\begin{figure}[H]
	\centering
	\caption{Cargas hidráulicas em \(m\) e linhas equipotenciais (h=11 \(m\) e h=12 \(m\))}
	\includegraphics[width=0.75\linewidth]{result}	
	\label{result}	
\end{figure}


\begin{itemize}
	\item Variável secundária - Velocidade de fluxo
\end{itemize}


O calculo da velocidade de fluxo é determinada a partir da seguinte equação.

\begin{equation}
[v]=-[R][g]
\end{equation}

sendo \([g]=[B][q]\), temos:

\begin{equation}
[v]=-[R][B][q]
\end{equation}

\begin{equation}\label{velocidade}
[v]=-\begin{bmatrix}
k_x&0\\
0&k_y
\end{bmatrix}\cdot\frac{1}{\hspace{4pt}\biggr|\color{white}\begin{bmatrix}\color{black}
	1&\color{black}x_1&\color{black}y_1\\\color{black}1&\color{black}x_2&\color{black}y_2\\\color{black}1&\color{black}x_3&\color{black}y_3
	\color{white}\end{bmatrix}\color{black}\biggr|}\begin{bmatrix}
y_{23}&y_{31}&y_{12}\\x_{32}&x_{13}&x_{21}
\end{bmatrix}\cdot \begin{bmatrix}
h_1\\
h_2\\
h_3
\end{bmatrix}
\end{equation}

 Subtituindo os correspondentes valores na equação \ref{velocidade} para cada elemento temos os correspondentes vetores de velocidade.
 
 
 \begin{equation}
 [v]^{1}=\begin{bmatrix}
1,30\\
0,00
 \end{bmatrix}\cdot 10^{-7}m/s
 \end{equation}
 
 \begin{equation}
[v]^{2}=\begin{bmatrix}
0,98\\
-0,33
\end{bmatrix}\cdot 10^{-7}m/s
\end{equation}

 \begin{equation}
[v]^{3}=\begin{bmatrix}
1,90\\
-0,33
\end{bmatrix}\cdot 10^{-7}m/s
\end{equation}

 \begin{equation}
[v]^{4}=\begin{bmatrix}
1,63\\
-0,60
\end{bmatrix}\cdot 10^{-7}m/s
\end{equation}

 \begin{equation}
[v]^{5}=\begin{bmatrix}
3,05\\
-0,60
\end{bmatrix}\cdot 10^{-7}m/s
\end{equation}

 \begin{equation}
[v]^{6}=\begin{bmatrix}
3,65\\
0,00
\end{bmatrix}\cdot 10^{-7}m/s
\end{equation}

 \begin{equation}
[v]^{7}=\begin{bmatrix}
0,98\\
0,00
\end{bmatrix}\cdot 10^{-7}m/s
\end{equation}
 \begin{equation}
[v]^{8}=\begin{bmatrix}
0,00\\
-0,98
\end{bmatrix}\cdot 10^{-7}m/s
\end{equation}
 \begin{equation}
[v]^{9}=\begin{bmatrix}
1,63\\
-0,98
\end{bmatrix}\cdot 10^{-7}m/s
\end{equation}

 \begin{equation}
[v]^{10}=\begin{bmatrix}
0,00\\
-2,60
\end{bmatrix}\cdot 10^{-7}m/s
\end{equation}

 \begin{equation}
[v]^{11}=\begin{bmatrix}
3,65\\
-2,60
\end{bmatrix}\cdot 10^{-7}m/s
\end{equation}

 \begin{equation}
[v]^{12}=\begin{bmatrix}
6,25\\
0,00
\end{bmatrix}\cdot 10^{-7}m/s
\end{equation}





\begin{figure}[H]
	\centering
	\caption{Velocidades nas direções x e y no meio poroso}
	\includegraphics[width=1\linewidth]{velocidade}	
	\label{velocidadefinal}	
\end{figure}






\begin{itemize}
	\item Efeito refinamento da malha
\end{itemize}

Nas seguintes figuras são mostrados a distribução de cargas hidraulicas e poropressões para uma malha com 12 elementos e 12 nós e para uma muito más refinada (678 elementos e 425 nós).

\begin{figure}[H]
	\centering
	\begin{subfigure}[b]{0.41\textwidth}
		\includegraphics[width=\textwidth]{option1_a}
		\caption{}
		\label{}
	\end{subfigure}\\
	\begin{subfigure}[b]{0.49\textwidth}
		\includegraphics[width=\textwidth]{option1_ahh}
		\caption{}
		\label{}
	\end{subfigure}
	\begin{subfigure}[b]{0.49\textwidth}
		\includegraphics[width=\textwidth]{option1_au}
		\caption{}
		\label{}
	\end{subfigure}
	\caption{(a)Discretização da malha com 12 elementos e 12 nós, (b) distribução de cargas hidraulicas (m) e (c) poropressões (kPa).}\label{}
\end{figure}

\begin{figure}[H]
	\centering
	\begin{subfigure}[b]{0.41\textwidth}
		\includegraphics[width=\textwidth]{option1_b}
		\caption{}
		\label{}
	\end{subfigure}\\
	\begin{subfigure}[b]{0.49\textwidth}
		\includegraphics[width=\textwidth]{option1_bhh}
		\caption{}
		\label{}
	\end{subfigure}
	\begin{subfigure}[b]{0.49\textwidth}
		\includegraphics[width=\textwidth]{option1_bu}
		\caption{}
		\label{}
	\end{subfigure}
	\caption{(a)Discretização da malha com 768 elementos e 425 nós, (b) distribução de cargas hidraulicas (m) e (c) poropressões (kPa).}\label{}
\end{figure}

\begin{itemize}
	\item Efeito forma da malha
\end{itemize}

\begin{figure}[H]
	\centering
		\begin{subfigure}[b]{0.44\textwidth}
		\includegraphics[width=\textwidth]{option1_am}
		\label{}
	\end{subfigure}
	\begin{subfigure}[b]{0.44\textwidth}
		\includegraphics[width=\textwidth]{option2_am}

		\label{}
	\end{subfigure}\\
	\begin{subfigure}[b]{0.49\textwidth}
	\includegraphics[width=\textwidth]{option1_ahh}

	\label{}
\end{subfigure}
	\begin{subfigure}[b]{0.49\textwidth}
	\includegraphics[width=\textwidth]{option2_ahh}

	\label{}
\end{subfigure}\\

\begin{subfigure}[b]{0.49\textwidth}
	\includegraphics[width=\textwidth]{option1_au}
	\caption{}
	\label{}
\end{subfigure}
	\begin{subfigure}[b]{0.49\textwidth}
	\includegraphics[width=\textwidth]{option2_au}
	\caption{}
	\label{}
\end{subfigure}
	\caption{Discretização da malha com 12 elementos e 12 nós, distribução de cargas hidraulicas (\(m\)) e poropressões (\(kPa\)) para elementos triangulares (a) TIPO I e (b) TIPO II.}\label{}
\end{figure}












\begin{figure}[H]
	\centering
	\begin{subfigure}[b]{0.43\textwidth}
		\includegraphics[width=\textwidth]{case3am}
		\label{}
	\end{subfigure}
	\begin{subfigure}[b]{0.46\textwidth}
		\includegraphics[width=\textwidth]{case3bm}
		
		\label{}
	\end{subfigure}\\
	\begin{subfigure}[b]{0.49\textwidth}
		\includegraphics[width=\textwidth]{case3ah}
		
		\label{}
	\end{subfigure}
	\begin{subfigure}[b]{0.49\textwidth}
		\includegraphics[width=\textwidth]{case3bh}
		
		\label{}
	\end{subfigure}\\
	
	\begin{subfigure}[b]{0.49\textwidth}
		\includegraphics[width=\textwidth]{case3au}
		\caption{}
		\label{}
	\end{subfigure}
	\begin{subfigure}[b]{0.49\textwidth}
		\includegraphics[width=\textwidth]{case3bu}
		\caption{}
		\label{}
	\end{subfigure}
	\caption{Discretização da malha com 48 elementos e 35 nós, distribução de cargas hidraulicas (\(m\)) e poropressões (\(kPa\)) para elementos triangulares (a) TIPO I e (b) TIPO II.}\label{}
\end{figure}






%BIBLIOGRAPHY
%\bibliographystyle{apalike}
%\bibliography{bibliography}


\end{document}
